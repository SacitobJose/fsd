% Setup -------------------------------

\documentclass[a4paper]{report}
\usepackage[a4paper, total={6in, 10in}]{geometry}
\setcounter{secnumdepth}{3}
\setcounter{tocdepth}{3}

\usepackage{hyperref}

% Encoding
%--------------------------------------
\usepackage[T1]{fontenc}
\usepackage[utf8]{inputenc}
%--------------------------------------

% Portuguese-specific commands
%--------------------------------------
\usepackage[portuguese]{babel}
%--------------------------------------

% Hyphenation rules
%--------------------------------------
\usepackage{hyphenat}
%--------------------------------------

% Capa do relatório

\title{
	Fundamentos de Sistemas Distribuídos
	\\ \Large{\textbf{Trabalho Prático}}
	\\ -
	\\ Mestrado em Engenharia Informática
	\\ \large{Universidade do Minho}
	\\ Relatório
}
\author{
	\begin{tabular}{ll}
		\textbf{Grupo}
		\\\hline
		PG41080 & João Ribeiro Imperadeiro
		\\
		PG41081 & José Alberto Martins Boticas
		\\
		PG41091 & Nelson José Dias Teixeira
	\end{tabular}
}

\date{\today}

\begin{document}

\begin{titlepage}
    \maketitle
\end{titlepage}

% Resumo

\begin{abstract}
	Este trabalho prático tem por base a implementação de um sistema de troca de mensagens com persistência e ordenação. Para tal, à semelhança do que feito durante as aulas, foi utilizada a linguagem \textit{Java} (que é orientada aos objetos) por forma a tomar partido de algumas classes já existentes para este tipo de problemas, como por exemplo a classe \textit{Atomix}. De forma geral, o sistema pretendido pode ser descrito como semelhante ao \textit{Twitter} com alguns requisitos extra de correção.
\end{abstract}

% Índice

\tableofcontents

% Introdução

\chapter{Introdução} \label{intro}
\large{
	Neste projeto é requerida a elaboração de um sistema distribuído, semelhante ao \textit{Twitter}, baseado na troca de mensagens. Este sistema deve satisfazer alguns requisitos que foram impostos pelo docente da unidade curricular, que se listam de seguida.
	\begin{itemize}
		\item o sistema deve incluir um conjunto de servidores que se conhecem todos entre si. Estes não devem ter qualquer interação direta com o utilizador. Admite-se ainda a possibilidade de cada um dos servidores ser reiniciado, devendo garantir que o sistema continua operacional depois de todos os servidores estarem novamente a funcionar;
		\item o sistema deve incluir clientes que se ligam a qualquer um dos servidores. Admite-se também que o cliente possa ser reiniciado e posteriormente ligado a um novo servidor.
		\item admite-se que tanto os clientes como os servidores possam fazer uso de memória persistente.
		\item o cliente deve incluir uma interface rudimentar para interagir com o sistema. Nesta deve-se incluir as seguintes funcionalidades:
		\begin{itemize}
			\item permitir a publicação de uma mensagem etiquetada com um ou mais tópicos;
			\item indicar qual a lista de tópicos subscritos;
			\item obter as últimas 10 mensagens enviadas para os tópicos subscritos.
		\end{itemize}
		\item o conjunto de mensagens obtido por cada cliente em cada operação deve refletir uma visão causalmente coerente das operações realizadas em todo o sistema, por esse ou outros utilizadores.
	\end{itemize}
}

\chapter{Implementação}
	O sistema implementado tem como requisito a inclusão de um conjunto de servidores que se conhecem todos entre si e que não devem ter qualquer interação direta com o utilizador.
	Por isso, foi criada uma interface (Client.java) que trata de suavizar essa interação e cumpre os restantes requisitos apresentados anteriormente.

	Foi importante definir algumas diretivas antes de se passar à implementação do sistema. Vamos agora enumerar algumas das decisões tomadas, que serão posteriormente explicadas:
	\begin{itemize}
			\item Cada servidor contém toda a informação do sistema.
			\item Existe \textit{total order} entre os servidores.
			\item Os pedidos dos clientes são de dois tipos (bloqueantes e não bloqueantes):
			\begin{itemize}
				\item Os \textbf{pedidos bloquantes} são: enviar mensagens (tweets) e subscrever tópicos. 
				\item Os \textbf{pedidos não bloquantes} são: ver mensagens (tweets) e ver a lista dos tópicos subscritos.
			\end{itemize}
			\item O cliente não usa memória persistente, ao contrário dos servidores, onde são guardados todos os dados.
			\item Os servidores utilizam o protocolo \textit{Two-phase Commit} para garantir a própria consistência e a coerência do sistema.
		\end{itemize}

	\section{Servidor}

		\subsection{Inicialização}
		Quando é iniciado, o servidor procura uma base de dados. Se encontrar uma e ficheiro de logs, assume essa mesma base de dados e tenta aplicar as transações pendentes no ficheiro de logs, caso exista alguma.
		Caso não encontre uma base de dados, assume que está a ser criado um novo sistema, de raíz.

		Para a inicialização do primeiro servidor, deve ser executado o ficheiro \textbf{Twitter.java}, com a indicação do IP e da porta.
		Os restantes servidores devem ter a mesma indicação, para além da indicação do IP e porta de um outro servidor já inicializado, por forma a poderem juntar-se a este.

		Aquando da criação de novos servidores, que não o primeiro, é realizado um \textit{two-phase commit} para garantir que todos os outros servidores já criados estão a par do novo membro do \textit{cluster}.
		Isto permite que sejam adicionados novos servidores mais tarde.


		\subsection{Pedidos bloquantes vs. não bloquantes}

		Os pedidos que o cliente pode fazer a um servidor do sistema foram divididos em dois tipos: 
		\begin{itemize}
			\item bloquantes - enviar mensagens e subscrever tópicos;
			\item não bloquantes - ver mensagens e ver a lista de tópicos subscritos.
		\end{itemize}
		
		Um pedido bloquante implica que o servidor tenha de comunicar o mesmo aos restantes servidores, iniciar um two phase commit e só depois responder ao cliente.
		Afirmativamente quando todos os servidores estão disponíveis para receber a atualização, ou negativamente, caso contrário.

		Um pedido não bloqueante não implica que o servidor tenha de comunicar com os restantes servidores, podendo desde logo responder ao cliente.

		O que acontece se um dos servidores tiver uma falha e não tiver ainda sido reiniciado, quando outro recebe um pedido?
		Se o pedido que o servidor recebe for bloqueante, então o mesmo fica bloqueado à espera que o servidor que falhou seja reposto.
		Se o pedido for não bloquante, então o servidor responde de imediato, uma vez que não põe em causa a coerência do sistema.


		\subsection{Coerência/Ordenação}

		% Explicar melhor a implementação do total order
		Os servidores têm \textit{total order}, implementada através de um contador. 
		Isto garante que a resposta a pedidos não bloqueantes efetuados ao mesmo tempo em servidores diferentes é igual, quer em termos de conteúdo como em termos de ordenação.

		O facto de a ligação entre um servidor e um cliente utilizar o protocolo TCP, permite uma visão causalmente coerente (\textit{causal order}), 
		uma vez que é impossível que uma mensagem "ultrapasse" outra. Ou seja, se um cliente envia duas mensagens para o servidor, a primeira chegará ao destino em primeiro lugar e, portanto, será processada em primeiro lugar.


		\subsection{Gestão dos dados}

		A informação sobre o sistema, como os tópicos subscritos por um utilizador ou as mensagens partilhadas pelo mesmo, é guardada em todos os servidores.
		Cada servidor tem uma base de dados, um objeto DBHandler (definida no ficheiro \textbf{DBHandler.java}), que contém a seguinte informação:
		\begin{itemize}
			\item um conjunto (\textit{HashSet<Address>}) com os endereços de todos os outros servidores;
			\item uma lista (\textit{ArrayList<Tweet>}) de todos os tweets (mensagens enviadas por utilizadores), 
			sendo que um \textit{Tweet} é composto por uma string "username" (que identifica um utilizador), outra string "content" (com o conteúdo da mensagem/tweet)
			e ainda um conjunto (\textit{HashSet<String>}) de tópicos;
			\item um mapa (\textit{HashMap<String, ArrayList<Integer>>}) de tópicos para uma lista de inteiros, sendo que esta lista de inteiros corresponde aos índices dos tweets com esse tópico.
			\item um mapa (\textit{HashMap<String, HashSet<String>>}) de usernames para um conjunto de tópicos subscritos (pelo utilizador identidicado por esse username).
		\end{itemize}
		
		Deste modo, um servidor tem sempre, em memória volátil, toda informação existente, não havendo assim o atraso da leitura de disco.
		De realçar que os objetos DBHandler de cada servidor devem ser iguais para todos os servidores, uma vez que representam toda a informação do sistema.


		\subsection{\textit{Two-phase Commit}}

		Sempre que uma operação bloqueante é despoletada, todos os servidores recebem essa informação e, caso tudo esteja em conformidade, prosseguem à gravação da informação.
		Para isso, é usado o protocolo \textit{two-phase commit}. Portanto, todos os servidores inicializam um objeto TPCHandler, definido no ficheiro \textbf{TPCHandler.java}, 
		que guarda a seguinte informação:
		\begin{itemize}
		\item os endereços dos outros servidores, para comunicar com os mesmos;
		\item a base de dados (objeto DBHandler), para ser alterada;
		\item um contador, para garantir \textit{total order};
		\item dois logs (um de servidor e um de coordenador), que permitem reiniciar uma transação no caso de um servidor falhar a meio da mesma;
		\item um mapa com os os tcps de coordenador  % EXPLICAR O QUE É ISTO
		\item um mapa de endereços (de cada um dos outros servidores) para outro mapa de inteiros para pares estado da transação/transação 
		(\textit{HashMap<Address, TreeMap<Integer, Pair<TPCStatus, TwoPhaseCommit>>>}), que corresponde às transações por completar.
		\end{itemize}
		
		Mas como é controlado o \textit{two-phase commit}?
		Neste sistema, por não ser implementada tolerância a faltas (não é objeto de estudo na UC), foi decidido, por razões de simplicidade, 
		que, numa transação, o coordenador é sempre o servidor que recebe o pedido e, consequentemente inicia a mesma.
		Assim, este é quem controla o \textit{two-phase commit}.

		O coordenador envia a informação a todos os servidores (incluindo ele próprio) e pergunta se todos podem guardar a informação. 
		No caso de todos os servidores darem resposta positiva, o coordenador indica-lhes que devem guardar a informação na base de dados (memória persistente), sendo a transação bem sucedida.

		% INCLUIR RAZÃO DE EXISTIREM OS LOGS

		Tal como os pedidos bloqueantes, a junção de um servidor também inicia um \textit{two-phase commit}, por ser uma operação relevante 
		(a partir daqui todas as operações bloqueantes devem passar por este novo servidor) e que implica a alteração da base de dados dos servidores.
		Se o \textit{two-phase commit} for bem sucedido, o novo servidor recebe todas as informações do sistema, por forma inserir-se no mesmo e poder dar uma resposta consistente aos clientes.

		Importa ainda referir que foi necessário definir um tipo diferente de \textit{two-phase commit}, os heartbeats. % EXPLICAR ISTO QUE EU NÃO CONSIGO


	\section{Cliente}
		Um servidor pode ser ligado a qualquer servidor, sendo que pode ainda desconectar-se e ligar-se a um outro servidor, sem perder os seus dados ou a ordenação dos mesmos.
		Para ser identificado, o utilizador fornece o seu username quando se conecta ao servidor. Se o username não existir na base de dados, é criado. Se já existir, assume as informações já existentes.

		Não foi implementada autenticação de utilizadores, por não ser relevante para a prática e demonstração das capacidades adquiridas na UC, pelo que não existe qualquer login e/ou registo de utilizadores.

		Com \textit{Two-phase Commit} e \textit{total order}, é assegurado que as respostas que o cliente recebe refletem uma visão consistente do estado de todos os servidores.


		\subsection{A interface}
		Um utilizador pode tomar 6 ações através da interface que lhe é disponibilizada:
		\begin{itemize}
			\item Tweet - publicar uma mensagem (tweet);
			\item Subscribe - subscrever tópicos, mantendo os que já subscreve;
			\item View subscriptions - visualizar quais os tópicos que subscreve;
			\item Last 10 from all subscribed topics - visualizar os últimos 10 tweets sobre os tópicos que subscreve;
			\item Last 10 from specific topics - visualizar os últimos 10 tweets sobre um conjunto de tópicos específico, a definir;
			\item Exit - terminar a conexão ao servidor/sistema.
		\end{itemize}

	\section{Trabalho extra}
		\subsection{Novos servidores}

		O sistema foi implementado de forma a que possam ser adicionados servidores ao mesmo depois deste já ter sido utilizado, 
		permitindo assim a alteração do número de servidores a qualquer momento no tempo e não ficando o sistema restringido ao número de servidores com que foi inicializado.
		Não foi implementada a possibilidade de um servidor ser desconectar do sistema, embora a mesma seja fácil (um simples two phase commit).


		\subsection{Modularização de código}
		% NÃO SEI O QUE ISTO É
	
\chapter{Testes}
	\section{Testes de funcionamento}
		O sistema foi testado localmente (uma única máquina e vários processos), por forma a garantir que o mesmo funcionava corretamente e que todos os requisitos eram cumpridos.

		Para além destes primeiros testes, foram realizados ainda testes adicionais em máquinas remotas (com recurso à plataforma DigitalOcean). Estas máquinas estavam espalhadas por diferentes cidades, países e continentes, 
		nomeadamente em Londres (Inglaterra), Nova Iorque (EUA), Toronto (Canadá), Frankfurt (Alemanha), Singapura e na Índia.
		Nestas máquinas, foram criados três servidores e alguns clientes. 
		Pudemos conferir que o sistema funcionava corretamente também nestas condições, com os servidores, espalhados pelos locais enunciados, a conectarem-se sem problemas.
		Confirmamos ainda que os clientes conseguiam performar todas as ações disponibilizadas pela interface de utilizador e que o sistema mantinha a sua consistência e coerência, 
		bem como tinha um comportamento correta em todas as circunstâncias. Para isso, foi simulada a falha de um e de dois dos servidores e o sistema cumpriu o que fora por nós proposto.

	\section{Teste de performance} % POR FAZER

\chapter{Conclusão}
\large{
	Após a demonstração da abordagem adotada pelo grupo na implementaçao do sistema pedido dá-se por concluída a realização deste projeto. 
	Neste foi possível satisfazer todos os requisitos requeridos bem como acrescentar algumas funcionalidades extra que foram por nós entendidas como relevantes e/ou interessantes.

	Foi possível experimentar as dificuldades na implementação de um sistema distribuído, nomeadamente de garantir a coerência, consistência e estabilidade do mesmo.
	Pudemos também perceber que o protocolo \textit{two-phase commit} permite dar passos importantes no alcance da consistência.

	Para além disso, foi possível aprofundar o conhecimento de alguns dos aspetos relativos à componente prática desta unidade curricular, melhorando, assim, 
	a nossa capacidade na implementação de sistemas que abordam esta filosofia computacional.
}

\appendix
\chapter{Observações}
\begin{itemize}
	\item Biblioteca \textit{Atomix}:
	\par \textit{\url{https://atomix.io/}}
	\item Documentação \textit{Java}:
	\par \textit{\url{https://docs.oracle.com/en/java/javase/11/docs/api/index.html}}
\end{itemize}


\end{document}